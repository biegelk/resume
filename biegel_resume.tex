%----------------------------------------------------------------------------------------
%	PACKAGES AND OTHER DOCUMENT CONFIGURATIONS
%----------------------------------------------------------------------------------------
\dshf
\documentclass{resume} % Use the custom resume.cls style

\usepackage{hyperref}
\usepackage[left=0.75in,top=0.6in,right=0.75in,bottom=0.6in]{geometry} % Document margins

\name{Katie Biegel} % Your name
\address{(480)~$\cdot$~298~$\cdot$~9444 \\ kbiegel@anl.gov \\ github.com/biegelk} % Your phone number and email

\begin{document}

%----------------------------------------------------------------------------------------
%	WORK EXPERIENCE SECTION
%----------------------------------------------------------------------------------------

\begin{rSection}{Work Experience}

\begin{rSubsection}{PowerAdvocate, Inc}{July 2015 - July 2017}{Energy Consulting Manager}{Boston, MA}
\item Lead analyst on project for price/strategic behavioral modeling of oilfield services firms in response to commodity prices, macroeconomic indicators, petroleum/natural gas extraction statistics, and other market forces
\begin{list}{}
\item Led team of four people on \$100k pilot engagement which resulted in expanded client relationship
\item Determined optimal multivariate regression modeling methodology and instructed team in statistical methods
\item Conducted extensive qualitative and quantitative market research to identify potential expected behavior patterns for further research and validate our results against observed industry trends
\end{list}
\item Developed comprehensive decision-making priority matrix for client considering replacing long-time, poorly-performing construction partner firm for capital asset build-out program of nearly \$10B over 10 years
\begin{list}{}
\item Self-studied discounted cash flow valuation method and wrote valuation model of client firm, incorporating variable cost impacts from utilization of original and candidate construction partners
\item Researched regulatory and rate-setting processes in client's jurisdiction in order to estimate impacts of high construction cost unpredictability on regulatory burden
\item Authored evidentiary documents in defense of this and other work performed for the same client, which was submitted in support of client's rate case hearing; successfully disproved faulty opposing arguments and persuaded regulatory commission to decide in client's favor on points in question
\end{list}
\end{rSubsection}

%------------------------------------------------

\begin{rSubsection}{Massachusetts Institute of Technology, Prof. Scott Kemp}{June 2013 - June 2015}{Undergraduate Researcher}{Cambridge, MA}
\item Developed MATLAB risk scenario model of new nuclear power plant construction projects to calculate the necessary expected profits from electricity sales needed to justify investment, considering project characteristics such as plant design, utility credit rating, and a historically-accurate spectrum of construction delays
\end{rSubsection}

\end{rSection}


%----------------------------------------------------------------------------------------
%	EDUCATION SECTION
%----------------------------------------------------------------------------------------

\begin{rSection}{Education}

{\bf University of Wisconsin Madison} \hfill {\em anticipated: June 2023} \\
Ph.D. Student in Nuclear Engineering \\
Research field: Agent-based modeling of electric power utility capital investment decisions
\\

{\bf Massachusetts Institute of Technology} \hfill {\em June 2015} \\ 
B.S. in Nuclear Science and Engineering \\
Research field: Construction project financial evaluation and risk management \smallskip \\

\end{rSection}





%----------------------------------------------------------------------------------------
%	SKILLS SECTION
%----------------------------------------------------------------------------------------

\begin{rSection}{Skills}

\begin{tabular}{ @{} >{\bfseries}l @{\hspace{6ex}} l }
Programming & Python; Julia; bash; git; Linux; SQLite \\
MS Excel & modeling; data analysis; templating/UI/entry validation \\
MS Powerpoint & technical/executive/layperson communication; infographics \\
\end{tabular}

\end{rSection}

%----------------------------------------------------------------------------------------
%	EXAMPLE SECTION
%----------------------------------------------------------------------------------------

%\begin{rSection}{Section Name}

%Section content\ldots

%\end{rSection}

%----------------------------------------------------------------------------------------

\end{document}
