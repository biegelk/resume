%----------------------------------------------------------------------------------------
%	PACKAGES AND OTHER DOCUMENT CONFIGURATIONS
%----------------------------------------------------------------------------------------

\documentclass{resume} % Use the custom resume.cls style

\usepackage{hyperref}
\usepackage[left=0.75in,top=0.5in,right=0.75in,bottom=0.6in]{geometry} % Document margins

\name{K a t i e \:\: B i e g e l} % Your name
\address{(480)~$\cdot$~298~$\cdot$~9444 \\ kbiegel@anl.gov \\ github.com/biegelk} % Your phone number and email

\begin{document}

%----------------------------------------------------------------------------------------
%	WORK EXPERIENCE SECTION
%----------------------------------------------------------------------------------------

\begin{rSection}{Work Experience}

\begin{rSubsection}{Graduate Researcher}{September 2018 -- present}{Argonne National Laboratory}{Lemont, IL}
  \setlength{\itemsep}{-5pt}
  \item Lead developer for the Agent-Based Capacity Expansion (ABCE) code
    \begin{itemize}
      \item Simulates asset-investment decision-making behavior of independent utility companies, each seeking to optimize its own portfolio's valuation subject to imperfect information and limited/evolving financial resources
      \item Explicit time-evolution: agents learn new information and produce new forecasts each year; construction projects and financial instrument issuances play out in real time; and detailed simulations of generation assets' market performance change every year based on system portfolio composition
      \item Main program structure in Python; agent decision algorithms in Julia; all data generated by system and agents saved to a SQLite database
      \item Developed flexible data structure and interface to allow ABCE to dynamically construct and run instances of external codes as market dispatch solver engines
      \item Currently updating ABCE data interface to facilitate adoption by interested research groups outside of ANL
    \end{itemize}

  \item Technical Analyst for the Civil Nuclear Credit (CNC) Program: developed an Excel operating profit model for all U.S. commercial nuclear reactor units, including impact of state and federal subsidy programs
  \begin{itemize}
    \item Researched and modeled state nuclear subsidy programs (CT, NY, NJ, IL) and federal production tax credit introduced in the Inflation Reduction Act (2022): structure, compensation levels, terms and conditions
    \item Developed models for revenue streams and operating costs based on best available information for each individual unit, and conducted sensitivity analysis on assumptions to provide context and more robust conclusions
    \item Wrote 12-page analysis summary report to CNC leadership's brief, to assist with preparations for the first CNC award cycle
    \item Model based solely on publicly-available information
  \end{itemize}

  \item Technoeconomics analyst: techhoeconomic decision factors and impacts of transitioning coal power plants (CPPs) into nuclear power plants (NPPs)
  \begin{itemize}
    \item Developed exemplar project schedules for all project types to model the critical economic impacts of the "revenue gap" (unavoidable zero-revenue period between CPP shutdown and NPP startup on the same site, mandatory in some projects due to equipment refurbishment and/or site remediation)
    \item Developed models of activities and costs required to decommission CPPs and manage coal waste facilities
    \item Used ABCE to model decision drivers and utility firms' preferences among possible types of CPP-to-NPP conversion projects
    \item Presented results to DOE leadership (August 2022) and at an ANS Winter Meeting panel session (November 2022)
  \end{itemize}

  \item Other projects and publications
  % TODO: get dates and links for all of these
  \begin{itemize}
    \item Impact of electric vehicle power demand and charging schedules on electricity markets and nuclear generators
    \item February 2021 Electricity Blackouts and Natural Gas Shortages in Texas
    \item Economic impacts of nuclear power plant load-following capabilities
    \item Report on reuse of decommissioned nuclear assets
  \end{itemize}

\end{rSubsection}

\begin{rSubsection}{Energy Consulting Manager}{July 2015 - July 2017}{PowerAdvocate, Inc.}{Boston, MA}
  \setlength{\itemsep}{-5pt}
  \item Lead analyst on project for price/strategic behavioral modeling of oilfield services firms in response to commodity prices, macroeconomic indicators, petroleum/natural gas extraction statistics, and other market forces
  \begin{itemize}
    \item Led team of four people on \$100k pilot engagement which resulted in expanded client relationship
    \item Determined optimal multivariate regression modeling methodology and instructed team in statistical methods
    \item Conducted extensive qualitative and quantitative market research to identify potential expected behavior patterns for further research and validate our results against observed industry trends
  \end{itemize}
  \item Developed comprehensive decision-making priority matrix for client considering replacing long-time, poorly-performing construction partner firm for capital asset build-out program of nearly \$10B over 10 years
  \begin{itemize}
    \item Self-studied discounted cash flow valuation method and wrote valuation model of client firm, incorporating variable cost impacts from utilization of original and candidate construction partners
    \item Researched regulatory and rate-setting processes in client's jurisdiction in order to estimate impacts of high construction cost unpredictability on regulatory burden
    \item Authored evidentiary documents in defense of this and other work performed for the same client, which was submitted in support of client's rate case hearing; successfully disproved faulty opposing arguments and persuaded regulatory commission to decide in client's favor on points in question
  \end{itemize}
\end{rSubsection}

%------------------------------------------------

\begin{rSubsection}{Undergraduate Researcher}{June 2013 - June 2015}{Massachusetts Institute of Technology, Prof. Scott Kemp}{Cambridge, MA}
  \setlength{\itemsep}{-5pt}
  \item Developed MATLAB risk scenario model of new nuclear power plant construction projects to calculate the necessary expected profits from electricity sales needed to justify investment, considering project characteristics such as plant design, utility credit rating, and a historically-accurate spectrum of construction delays
\end{rSubsection}

\end{rSection}


%----------------------------------------------------------------------------------------
%	EDUCATION SECTION
%----------------------------------------------------------------------------------------

\begin{rSection}{Education}

\begin{rSubsection}{Doctoral Candidate in Nuclear Engineering}{anticipated: June 2023}{University of Wisconsin -- Madison, Department of Engineering Physics}{Madison, WI}
  \item Research field: Agent-based modeling of electric power utility capital investment decisions
\end{rSubsection}

\begin{rSubsection}{Bachelor of Science in Nuclear Science and Engineering}{June 2015}{Massachusetts Institute of Technology, Department of Nuclear Science and Engineering}{Cambridge, MA}
  \item Senior thesis: construction project financial evaluation and risk management
\end{rSubsection}

\end{rSection}





%----------------------------------------------------------------------------------------
%	SKILLS SECTION
%----------------------------------------------------------------------------------------

\begin{rSection}{Skills}

\begin{tabular}{ @{} >{\large}l @{\hspace{6ex}} l }
Programming & Python; Julia; bash; git; Linux; SQLite \\
MS Excel & modeling; data analysis; templating/UI/entry validation \\
MS Powerpoint & technical/executive/layperson communication; infographics \\
\end{tabular}

\end{rSection}

%----------------------------------------------------------------------------------------
%	EXAMPLE SECTION
%----------------------------------------------------------------------------------------

%\begin{rSection}{Section Name}

%Section content\ldots

%\end{rSection}

%----------------------------------------------------------------------------------------

\end{document}
